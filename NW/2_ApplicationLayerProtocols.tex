\section{Application layer protocols}
\enumstart
	\item Basic concepts
	\enumstart
		\item Communication of applications running on different hosts
		\item Client
		\enumstart
			\item Initiates contact with server
			\item Requests service from server
		\enumend
		\item Server
		\enumstart
			\item Provides requested service to client
		\enumend
		\item Application programming interface (API)
		\enumstart
			\item Defines interface between application and transport layer
			\item Two processes communicate by sending data into socket, reading data from socket
		\enumend
		\item Needed transport services
		\enumstart
			\item Data loss tolerance
			\item Timing requirements
			\item Bandwidth requirements
		\enumend
	\enumend
	\item Transmission control protocol (TCP)
	\enumstart
		\item Connection oriented \arrow setup required between client and server
		\item Reliable data transfer
		\item Flow control
		\item Congestion control \arrow throttle sender when network overloaded
		 \item No timing and minimum bandwidth guarantee
	\enumend
	\item User datagram protocol (UDP)
	\enumstart
		\item Unreliable data transfer
		\item No connection setup, reliability, flow control, congestion control, timing or bandwidth guarantee
	\enumend
	\item Hypertext transfer protocol (HTTP)
	\enumstart
		\item step-by-step
		\enumstart
			\item Clients indicate TCP connection (typically on port 80)
			\item Server accepts TCP connection
			\item HTTP messages exchanged between browser and webserver
			\item TCP connection is closed
		\enumend
		\item HTTP/1.0
		\enumstart
			\item Non-persistent
			\item New TCP connection for each file to transmit \arrow slow
		\enumend
		\item HTTP/1.1
		\enumstart
			\item Persistent
			\item Multiple files in one TCP connection \arrow faster
		\enumend
		\item HTTP request
		\enumstart
			\item Encoded in ASCII
			\item Carriage return and line feed indicate end of message
			\\ \customImage{http_request.png}{0.25}
		\enumend
		\item HTTP request methods
		\enumstart
			\item GET
			\item POST
		\enumend
		\item HTTP response
		\enumstart
			\item Encoded in ASCII
			\\ \customImage{http_response.png}{0.3}
		\enumend
		\item HTTP response status codes
		\enumstart
			\item 200 OK
			\item 301 Moved permanently
			\item 400 Bad request
			\item 404 Not found
			\item 505 HTTP version not supported
		\enumend
	\enumend
	\item Telnet (Remote terminal access)
	\enumstart
		\item Typically on port 23
		\item No security encryption
		\item Superceded by SSH
		\item Useful for debugging text-based protocols (HTTP, FTP, SMTP, POP, \ddd)
	\enumend
	\item Cookies in HTTP
	\enumstart
		\item Server sends cookie to client in response message
		\item Client presents cookie in later request
		\\ \customImage{cookie.png}{0.45}
		\item Expires: when to throw this cookie away
		\item Domain: who to present this cookie to
		\item Path: which URL to present this cookie to
		\item The rest: page dependant
	\enumend
	\item Caching (websites are cached in various layers \arrow faster)
	\item Domain name system (DNS)
	\enumstart
		\item Maps domain names to IP-Addresses
		\item Distributed database (hierarchical)
		\item Application layer protocol
		\item Types
		\enumstart
			\item Root name server (maps TLDs to their IP-addresses)
			\item Authoritative name server (store IP-addresses and names of all nodes in it's zone)
			\item Local name server (e.g. for a company)
			\item DNS resolver (local to the application)
		\enumend
		\item Recursive search: ask DNS server that is responsible for the best suffix you know after an IP adress for a better suffix \arrow until domain name is found
		\item Hierarchical search: ask your DNS server to find the domain name for you \arrow he does the same
		\item DNS caching \arrow on all levels
		\item Ressource records
		\enumstart
			\item Format: (name, ttl, class, type, value)
			\item Types
			\enumstart
				\item A (name is hostname, value is IP address)
				\item CNAME (name is alias name for some canonical name, value is canonical name)
				\item NS (name is domain, value is IP address of authoritative name server for this domain)
				\item MX (value is mail server assiciated with name)
			\enumend
		\enumend
		\item DNS protocol messages
		\enumstart
			\item Typically on port 53
			\item Query and reply messages, both with the same message format
			\item Reply uses the same identification number as query
			\\ \customImage{dns_message.png}{0.3}
			\item Flags
			\enumstart
				\item Query or reply
				\item Recursion desired
				\item Recursion available
				\item Reply is authoritative
			\enumend
		\enumend
	\enumend
	\item Domain name
	\enumstart
		\item Top level domain (TLD)
		\enumstart
			\item Two letters are always country-TLDs (e.g. .ch, .us, .uk, \ddd)
			\item Three letters are special TLDs (e.g. .com, .org, .edu, .mil, \ddd)
		\enumend
		\item Subdomains
		\enumstart
			\item Any TLD or subdomain $x$, can have any subdomain $y.x$ \arrow tree structure
		\enumend
	\enumend
	\item Simple mail transfer protocol (SMTP)
	\enumstart
		\item Typically TCP on port 25
		\item Direct transfer from sending server to receiving server
		\item 3 phases
		\enumstart
			\item Handshake
			\item Transfer of messages
			\item Closure
		\enumend
		\item 7-bit ASCII text based
		\item Persistent connections
		\item Carriage return - line feed - carriage return - line feed \arrow end of message
	\enumend
\enumend
