\section{Network layer}
\enumstart
	\item Transport packet from sending to receiving host (NIC)
	\item Function
	\enumstart
		\item Path determination (routing algorithm)
		\item Switching (from routers input to routers output)
		\item Call setup (some network architectures require router call setup along path)
	\enumend
	\item Service models
	\enumstart
		\item Virtual circuits
		\enumstart
			\item Path tries to behave like a physical circuit
			\item Call setup
			\item Packet carries VC identifier
			\item Link, router ressources may be allocated to VC
			\item Used in ATM, frame-relay, X.25 (not used in todays internet)
		\enumend
		\item Datagram
		\enumstart
			\item No call setup at network layer
			\item Routers keep no state about end-to-end connections
			\item Packets routet using destination host ID (different paths)
		\enumend
	\enumend
	\item Routing
	\enumstart
		\item Series of local routing decisions (routing tables)
		\item Routing table is gathered by using a routing protocol
	\enumend
	\item Internet protocol (IP)
	\enumstart
		\item Address: IP address
		\item Datagram based
		\enumstart
			\item Best effort, unreliable
			\item Simple routers
			\item Packet fragmentation and reassembly
		\enumend
		\item Packet format
		\enumstart
			\item Version (4 bit)
			\item Header length (4 bit)
			\item Type of service (8 bit)
			\item Length (16 bit)
			\item Identifier (16 bit)
			\item Flags (3 bit)
			\item Offset (13 bit)
			\item Time to live (8 bit)
			\item Upper layer protocol (8 bit)
			\item Header checksum (16 bit)
			\item Source address (32 bit)
			\item Destination address (32 bit)
			\item Options (optional) (n $\times$ 32 bit, maximal 40 byte)
		\enumend
		\item Fragmentation and reassembly
		\enumstart
			\item Every network has a maximum transfer unit (MTU)
			\item Fragments reassemble only at destination
			\item Identifier, flag and offset are used to mark fragments and reassemble thems
			\item Fragmentation happens in addition to fragmentation at the transport layer
		\enumend
		\item IP addresses
		\enumstart
			\item 32 bit, split into network part and host part
			\item Class-full addressing (not used anymore, stupid)
			\item Classless interdomain routing (CIDR) (netmask replaces class)
			\item IP address is either manually configured or delivered by DHCP server
			\item Dynamic host configuration protocol (DHCP)
			\enumstart
				\item Host: DHCP discover
				\item Server: DHCP offer
				\item Host: DHCP request
				\item Server: DHCP ack
			\enumend
			\item How does an ISP get a block of IP addresses? Internet Corporation for Assigned Names and Numbers (ICANN)
		\enumend
	\enumend
	\item Internet control message protocol (ICMP)
	\enumstart
		\item Used by hosts, routers, gateways to exchange network-level information
		\item e.g. ping
	\enumend
	\item Network address translation
	\enumstart
		\item Translates the packets the router gets to the correct destination in the LAN
		\item Keeps state of the connections to assign packets to hosts
		\item Servers cannot initially connect to a host in the LAN (which host is meant?)
		\item Controversial
		\enumstart
			\item Routers should only process up to Layer 3
			\item Violates end-to-end argument (is like a man in the middle)
			\item IPv6 should be used
		\enumend
	\enumend
	\item Routing
	\enumstart
		\item Goal: determine good path through network
		\item Routing protocol classes
		\enumstart
			\item Distance vector protocol
			\enumstart
				\item Nodes know only distance (cost) to neighbors
				\item Exchange distance to all nodes with neighbors
				\item Update local information based on received information
			\enumend
			\item Link state protocol
			\enumstart
				\item All nodes know network topology
				\item Run algorithm to find shortest path (graph algorithm)
			\enumend
			\item Convergence: how fast until it stabilizes/reacts to changes?
			\item Security: Can malicious routers game the network
		\enumend
		\item Distance vector protocols
		\enumstart
			\item Algorithm is iterative, asynchronous, distributed
			\item $D^x(y) = min_z(D^x(y,z))$
			\item $D^x(y,z) = c(x,z) + D^z(y)$
			\item Local iteration caused by
			\enumstart
				\item Local link cost change
				\item Neighbor sends a message saying that one of its least cost paths changed
				\item Each node notifies neighbors, when local link cost changes
			\enumend
			\item Count to infinity problem (
			\item Routing information protocol (RIP)
			\enumstart
				\item Distance metric: number of hops
				\item Distance vectors: exchanged every 30 seconds via response messages (aka advertisements)
				\item If no advertisement from neighbor after 180 seconds \arrow neighbor is dead
				\item Poisoned reverse: if Z routes through Y to get to X, Z tells Y it's distance to X is infinite
				\item Advertisements in UDP packets
			\enumend
			\item $[$Enhanced$]$ interior gateway routing protocol ($[$E$]$IGRP)
			\enumstart
				\item CISCO proprietary
				\item Several cost metrics: delay, bandwidth, reliability, load \ddd
				\item Uses TCP to exchange routing updates
				\item Loop free routing via distributed updating algorithm (DUAL)
			\enumend
		\enumend
		\item Link state routing protocols
		\enumstart
			\item Every node knows the topology and cost of every link
			\enumstart
				\item Achieved through flooding
				\item Nodes send information on their links and neighbors to all neighbors
				\item Nodes forward information about other nodes to their neighbors
				\item ACK used to prevent message loss
				\item Sequence number used to compare versions
			\enumend
			\item With this information: Calculate shortest path to every possible destination
			\item Open shortest path first (OSPF)
			\enumstart
				\item Publicly available
				\item Advertisement carries one entry per neighbors router
				\item All OSPF messages authenticated
				\item Multiple same cost paths allowed
				\item Integrated uni- and multicast support
			\enumend
		\enumend
	\enumend
	\item IPv6
	\enumstart
		\item 128 bit addresses (32 bit was not enough anymore)
		\item Anycast
		\item No fragmentation allowed
		\item Fixed length 40 byte header
		\enumstart
			\item Version (4 bit)
			\item Traffic class (8 bit)
			\item Flow label (20 bit)
			\item Payload length (16 bit)
			\item Next header (8 bit)
			\item Hop limit (8 bit)
			\item Source address (128 bit)
			\item Destination address (128 bit)
		\enumend
		\item No checksum
		\item ICMPv6
		\item From IPv4 to IPv6
		\enumstart
			\item Dual Stack
			\item Tunneling
		\enumend
	\enumend
\enumend
