\section{Relational data model}
\enumstart
	\item Relation
	\enumstart
		\item $R \subseteq D_1 \times \ddd \times D_n$
		\item $D_1, \ddd, D_n$ are domains
	\enumend
	\item Tuple: $t \in R$
	\item Schema: associates labels to domains
	\item Instance: the state of the database
	\item Key: minimal set of attributes that identify each tuple uniquely
	\item Primary key: a randomly chosen key, used for references
	\item Step-by-step
	\enumstart
		\item Implementation of entities: (e.g. Table: $\{[$\underline{Legi: integer}, Name: string, Semester: integer$]\}$)
		\item Implementation of relationships: (e.g attends: $\{[$\underline{Legi: integer, Nr: integer}$]\}$)
		\enumstart
			\item If the ER specifies roles, use them for names
			\item If there are no roles, use the keys of the entities
			\item Otherwise invent something
		\enumend
		\item Merge relations with the same key
		\item Generalization: multiple possibilities to handle \arrow all are shitty
		\item Weak entities: 
	\enumend
\enumend

\subsection{Relational algebra}
\enumstart
	\item Symbols
	\enumstart
		\item $\sigma$ Selection
		\item $\Pi$ Projection
		\item $\times$ Cartesian product
		\item \join Join
		\item $\rho$ Rename
		\item $-$ Set minus
		\item $\div$ Relational division
		\item $\cup$ Union
		\item $\cap$ Intersection
		\item $\leftsemijoin$ Semi-join (left)
		\item $\rightsemijoin$ Semi-join (right)
		\item $\leftouterjoin$ Left outer-join
		\item $\rightouterjoin$ Right outer-join
	\enumend
	\item Atoms
	\enumstart
		\item A relation in the database
		\item A constant relation
	\enumend
	\item Operators
	\enumstart
		\item Selection: get subset of all tuples in a relation: (e.g. $\sigma_{\text{Semester }> 10}$(Student))
		\item Projection: get a subset of all attributes from all tuples in a relation (e.g. $\Pi_{\text{Level}}$(Professor))
		\item Cartesian product: get all combinations of two relations (e.g. Assistant $\times$ Professor)
		\item Natural join: R \join S = $\Pi_{\{y|y \in R\}\cup\{z|z \in S\}}$($\sigma_{\forall x,x \in R \land x \in S: \text{R.$x$ = S.$x$}}$(R $\times$ S))
		\item Theta join: super category of natural join, uses an arbitrary relation instead of the equality operator: R $\join_\theta$ S = $\sigma_{\theta}$(R $\times$ S)
		\item Left outer join: L $\leftouterjoin$ R = (L $A$ R) $\cup$ (L - $\Pi_{\text{L}}$(L $A$ R))
		\item Right outer join: L $\rightouterjoin$ R = (L $A$ R) $\cup$ (R - $\Pi_{\text{R}}$(L $A$ R))
		\item Full outer join:  L $\fullouterjoin$ R = (L $A$ R) $\cup$ (L - $\Pi_{\text{L}}$(L $A$ R)) $\cup$ (R - $\Pi_{\text{R}}$(L $A$ R))
		\item Left semi-join: (L $\leftsemijoin$ R) = $\Pi_{L}$(L $A$ R)
		\item Right semi-join: (L $\rightsemijoin$ R) = $\Pi_{R}$(L $A$ R)
		\item Rename: renaming of relation names (e.g. $\sigma_{bla}$($\rho_{L1}$(requires) $\times$ $\rho_{L2}$(requires)))
		\item Intersection: only with equal schemas: R $\cap$ S = R $-$ (R $-$ S)
		\item Relational division: R $\div$ S = $\Pi_{(R-S)}$(R) $-$ $\Pi_{(R-S)}$(($\Pi_{(R-S)}$(R) $\times$ S) $-$ R) 
	\enumend
\enumend

\subsection{Relational calculus}
\enumstart
	\item Queries have the form: $\{t \ | \ P(t)\}$ with $t$ a variable and $P(t)$ a predicate
	\item Two variants
	\enumstart
		\item Tuple relational calculus
		 \enumstart
		 	\item Atoms
		 	\enumstart
		 		\item $s | R$ ($s$ is a tuple, $R$ is a relation)
		 		\item $s.A \phi t.B$ or $s.A \phi c$ ($s,t$ tuple variables, $A,B$ attribute names, $\phi$ a comparison, $c$ a constant)
		 	\enumend
		 	\item Formulas
		 	\enumstart
		 		\item All atoms are legal formulas
		 		\item if $P$ is a formula, then $\lnot P$ and $(P)$ are formulas
		 		\item if $P_1$ and $P_2$ are formulas, then $P_1 \land P_2$, $P_1 \lor P_2$, $P_1 \implies P_2$ are formulas
		 		\item If $P(t)$ is a formula with a free variable $t$, then $\forall t \in R(P(t))$
		 	\enumend
		 	\item Safety: result must be a subset of the "domain of the formula"
		 	\item Domain: all constants in the formula + all domains of relations used in the formula
		 \enumend
		\item Domain relational calculus (variables iterate over domains)
		\enumstart
			\item Expressions: $\{[v_1, \ddd, v_n] \ | \ P(v_1, \ddd v_n)\}$
		\enumend
		\item Codds theorem: the following languages are equivalent
		\enumstart
			\item Relational algebra
			\item Tuple relational calculus (safe expressions)
			\item Domain relational calculus (safe expressions)
		\enumend
		\item \arrow SQL is based on relational calculus, SQL implementation is based on relational algebra
	\enumend
\enumend
