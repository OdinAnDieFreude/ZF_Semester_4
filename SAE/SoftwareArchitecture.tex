\section{Software Architecture}

\subsection{Cohesion}
Cohesion measures interdependance of the elements of one module. Higher cohesion is better.
\enumstart
	\item Operations work on the same data
	\item Operations implement a common abstraction (abstract data type)
\enumend

\subsection{Coupling}
Coupling measures interdependance between different modules. Lower coupling is better.
\enumstart
	\item Small interfaces
	\item information hiding
	\item No global data
	\item Interactions are within subsystems rather than across subsystem boundaries
\enumend

\subsection{Architectural Styles}

\subsubsection{Stream processing}
When large amounts of data are processed through several computational steps.
\enumstart
	\item Filter
	\enumstart
		\item Typically an autonomous unit of computation
		\item Does not touch shared memory
		\item read streams of input data, then locally transform input data and produce stream of output data
	\enumend
	\item Pipeline connector
	\item Splitjoin connector
	\enumstart
		\item Connect components in parallel
		\item Expose data parallelism and data distribution
		\item Spitter can be:
		\enumstart
			\item Round-robin - Each item to different child stream
			\item Duplicate - Data is dupplicated on output of splitter
		\enumend
		\item Joiner can be of various type (e.g. Round-robin)
	\enumend
	\item Strengths
	\enumstart
		\item Reuse - Any two filter can be connected if they agree on the data format
		\item Ease of maintenance - filters can be added or replaced
		\item Potential for parallelism
	\enumend
	\item Weaknesses
	\enumstart
		\item Sharing global data is expensive or limiting
		\item Can be difficult to design incremental filters
		\item Not appropriate for interactive applications
		\item Error handling is Achilles-heel
	\enumend
\enumend

\subsubsection{Event-based}
\enumstart
	\item Characterized by the style of communication between components
	\item Components may
	\enumstart
		\item generate events
		\item register for events of other components with a callback
	\enumend
	\item Connectors
	\enumstart
		\item Bindings between event generation and method calls (callbacks)
	\enumend
	\item Generators of events do not know  which  components will be affected by their events
	\item Components cannot make assumptions about the ordering in which events arise
	\item Strengths
	\enumstart
		\item Strong support for reuse in the large - Components can be introduced by simple registration
		\item Maintenance - Add and replace components with minimal effect on other components in the system
	\enumend
	\item Weaknesses
	\enumstart
		\item Loss of control
		\enumstart
			 \item What components will respond to an event?
			 \item In which order will components be invoked?
			 \item Are invoked components finished?
		\enumend
		\item Ensuring correctness is difficult, because it depends on context in which invoked
	\enumend
\enumend

\subsubsection{Call-and-return}
\enumstart
	\item Components: Objects
	\item Connections: Messages (method invocation)
	\item Key aspects
	\enumstart
		\item Object preserves integrity of representation (encapsulation)
		\item Representation is hidden from client objects
	\enumend
	\item Strengths
	\enumstart
		\item Change implementation without affecting clients
		\item Can break problems into interacting agents (distributed across multiple machines/networks)
	\enumend
	\item Weaknesses
	\enumstart
		\item Objects must know their interaction partner
		\item When partner changes, objects that explicitly invoke it must change
		\item Side effects
	\enumend
\enumend
