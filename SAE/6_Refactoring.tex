\section{Refactoring}
\enumstart
	\item Many programs are designed or implemented in a suboptimal way
	\enumstart
		\item Poor performance
		\item Hard to maintain
		\item Hard to extend
	\enumend
	\item Definition: Behaviour-preserving changes that improve the internal structure of existing code
	\item Why to refactor?
	\enumstart
		\item improves the design of the software
		\item makes code easier to understand
		\item helps to find bugs
		\item helps to program faster
	\enumend
	\item When to refactor?
	\enumstart
		\item When adding a new feature
		\item When fixing a bug
		\item When doing code review
	\enumend
	\item What to refactor?
	\enumstart
		\item Duplicate code
		\item Long methods
		\item Large classes
		\item Long or recurring switch statements
		\item changes that imply many other changes
	\enumend
	\item Discussion
	\enumstart
		\item Refactoring vs performance
		\enumstart
			\item Refactoring may cause slowdown
			\item But: for well-defined code, its much easier to tune performance
		\enumend
		\item Refactoring vs management
		\enumstart
			\item Effects of bad design show up after some time
			\item Refactoring enables long-term evolution
		\enumend
		\item Sometimes rewriting is easier than refactoring
	\enumend
\enumend

\subsection{Composing methods}
\enumstart
	\item Adapt the way code is packaged into methods
	\item Break large methods into smaller ones
	\item Combine small methods into a larger one
	\item Goal: Methods that are small enough to be understood easily
\enumend

\subsubsection{Extract method}
\enumstart
	\item Steps
	\enumstart
		\item Turn code fragment into a method whose name explains the purpose of the method
		\item Turn local variables used in the fragment into parameters of the method
		\item Turn local variables defined in the code fragment into the methods return value
	\enumend
	\item Benefits
	\enumstart
		\item Reuse: finer-graned methods are more likely to be reused
		\item Adaptability: Overriding the extracted code becomes easier
		\item Clarity: The higher-level method becomes easier to read
	\enumend
\enumend

\subsubsection{Inline methods}
\enumstart
	\item Given: A methods body is just as clear as its name
	\item Steps
	\enumstart
		\item Put the methods body into the body of the callers and remove the method
	\enumend
	\item Opposite of Extract method
	\item Apply only if method is not polymorphic
	\item Benefits
	\enumstart
		\item Clarity: Avoid useless indirection
		\item Enables other refactorings
	\enumend
\enumend

\subsubsection{Replace method with method object}
\enumstart
	\item Given: Large method that is difficult to decompose because of local variables
	\item Steps
	\enumstart
		\item Create a new class with a field for each local variable and each parameter of the method, and for the object that hosts the method
		\item Add constructor that initializes all fields
		\item Copy the original methods code into a new method compute()
		\item Replace the method with code that instantiates the new class and calls compute()
	\enumend
	\item Benefits
	\enumstart
		\item Enables other refactorings
		\item Can decompose a method, even if it uses many local variables
	\enumend
\enumend

\subsection{Moving features between objects}
\enumstart
	\item Where to put which responsibilities?
	\item Goal: improve assignments of fields and methods to classes
\enumend

\subsubsection{Move method}
\enumstart
	\item Given: A method using (or used by) more features of another class then its current class
	\item Steps
	\enumstart
		\item Check that superclasses and subclasses do not declare the method
		\item Move the method to the target class and adjust it - pass fields of source object or the source object itself
		\item Turn source method into a delegating method or remove it altogether
	\enumend
	\item Benefits
	\enumstart
		\item Reduce coupling of classes
		\item Shrink a large class by reducing its responsibilities
		\item Most usefull with other moves of fields and methods
	\enumend
\enumend

\subsubsection{Move field}
\enumstart
	\item Similar to "move method"
	\item Applicable if field is used by another class more than by its own class
	\item Create a new field in the target class and change all its users
\enumend

\subsubsection{Extract class}
\enumstart
	\item Given: A class does work that should be done by two
	\item Steps
	\enumstart
		\item Create a new class and reference it from the old class
		\item Use "move field" on each field
		\item Use "move method" on each method - start with lower level methods
	\enumend
	\item Benefits
	\enumstart
		\item Increased encapsulation
		\item Counteracts the tendency to add more and more features to classes
	\enumend
\enumend

\subsubsection{Inline class}
\enumstart
	\item Counterpart of "extract class"
	\item Apply if a class has few or no responsibilities
	\item Move all features into the other class and remove the old class
\enumend
	
\subsubsection{Hide delegate}
\enumstart
	\item Given: client calls a delegate class via a server object
	\item Steps
	\enumstart
		\item For each method of the delegate, create a delegating method on the server class
		\item Adjust the client to call the server
		\item Remove the servers method that returns the delegate (if not needed anymore)
	\enumend
	\item Benefits
	\enumstart
		\item Encapsulation - client becomes independent of how the server accesses the delegate
		\item Reduced coupling - delegate class is hidden from the client
	\enumend
\enumend

\subsubsection{Remove middle man}
\enumstart
	\item Counterpart of "hide delegate"
	\item If a class does too much delegation, let the client call the delegate directly
	\item Consequence: Avoids delegating methods, but reduces encapsulation
\enumend

\subsection{Organizing data}
\enumstart
	\item The best way to represent your data may change over time
	\item Goal: adapt data representation
\enumend

\subsubsection{Replace data with object}
\enumstart
	\item A data item that needs additional data or behaviour
	\item Steps
	\enumstart
		\item Create a class for the value, add a final field for the value
		\item Change the type of the field in the source class
		\item Let the getter in the source class call the getter of the new class
		\item Let the setter in the source class create a new instance of the new class
	\enumend
	\item Benefits
	\enumstart
		\item Enable adding fields and methods to the data item
		\item Deals with evolution, where a simple piece of data becomes more important over time
	\enumend
\enumend

\subsubsection{Replace array with object}
\enumstart
	\item Given: an array that stores data records
	\item Steps
	\enumstart
		\item Create a class and give it a public field for the array
		\item Change all users of the array to use the new class
		\item For each element of the array, replace direct array access with getters and setters
		\item Replace the array field with a field for each array element
	\enumend
	\item Benefits
	\enumstart
		\item Understandability
		\item Maintainability
		\item Encapsulation - Getters and setters may perform checks without influencing the client
		\item May add behaviour to the class later on
	\enumend
\enumend

\subsection{Simplifying conditional expressions}
\enumstart
	\item Conditional logic is tricky
	\item Goal: Simplify or remove conditional expressions
\enumend

\subsubsection{Decompose conditional}
\enumstart
	\item Given: a complicated conditional statement
	\item Steps
	\enumstart
		\item Extract condition into its own method
		\item Extract then-part and else-part into their own methods
	\enumend
	\item Benefit: Code becomes easier to unserstand and bugs become easier to find
\enumend

\subsubsection{Consolidate conditional expressions}
\enumstart
	\item Givem: sequence of conditional tests with the same result
	\item Steps
	\enumstart
		\item Check that all conditionals are side-effect free
		\item Replace sequence with a single conditional
		\item Consider extracting combined condition into a method
	\enumend
	\item Benefit: Code becomes easier to unserstand
\enumend

\subsubsection{Replace conditional with polymorphism}
\enumstart
	\item Given: conditional that chooses behaviour based on the type of an object
	\item Steps
	\enumstart
		\item Create a class hierarchy (simple subclasses or the strategy pattern)
		\item Use "move method" to bring the conditional to the top of the hierarchy
		\item For each subclass, move the code from the conditional into a method that overrides the superclass method
	\enumend
	\item Benefits
	\enumstart
		\item Particularly usefull if the same conditional appears in many places
		\item Adding a new type becomes easier because only one location must be changed
		\item Avoid reimplementing dynamic dispatch which is already part of the language
	\enumend
\enumend

\subsection{Making method calls simpler}
\enumstart
	\item API design is important
	\item Goal: make using an api easier
\enumend

\subsubsection{Replace parameters with explicit methods}
\enumstart
	\item Given: method runs different code depending on the value of an enumerated parameter
	\item Steps
	\enumstart
		\item Create a method for each value of the parameter
		\item For each case of the conditional, call the appropriate method
		\item Change all callers of the original method to use the new methods
		\item Remove the original method
	\enumend
	\item Benefits
	\enumstart
		\item Clarity - Can understand API from method names, without looking at parameters
		\item Safety - Compile-time checking instead of unsafe parameter that determines the behaviour
	\enumend
\enumend

\subsubsection{Introduce parameter object}
\enumstart
	\item Given: group of parameters that naturally go together
	\item Steps
	\enumstart
		\item Create a new, immutable class to represent the parameters
		\item Add a parameter of the new class to all methods with the group of parameters
		\item Adapt all callers by passing a new instance of the new class
		\item For each parameter, move the directly passed parameter into the constructor of the new class
	\enumend
	\item Benefits
	\enumstart
		\item Understandability - Reduces the length of the parameter list
		\item May add behaviour to the new class later on
	\enumend
\enumend

\subsection{Dealing with generalization}
\enumstart
	\item Several refactoring to refine the use of inheritance
	\item Moving features up/down the hierarchy
	\item Refining the hierarchy
\enumend
