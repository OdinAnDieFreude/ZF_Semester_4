\section{Requirements Engineering}
A Requirement is a feature that the System must have or a constraint it must satistfy to be accepted by the client.\\
Requirements define what a System must do/satisfy and not how.
Part of Requirements
\enumstart
	\item Functionality
	\item User interaction
	\item Error handling
	\item Environmental conditions (interfaces)
\enumend

\subsection{Functional Requirements}
\enumstart
	\item Functionality - What is the software supposed to do?
	\item External interfaces - Interaction with people, hardware or software
\enumend

\subsection{Nonfunctional Requirements}
\enumstart
	\item Performance - Speed, availability, response time, recovery time
	\item Attributes (quality requirements) - Portability, correctness, maintainability, security
	\item Design constraints - Required standards, operating environment
\enumend

\subsection{Functionality}
\enumstart
	\item Relationship of outputs to inputs
	\item Response to abnormal situations
	\item Exact sequence of operations
	\item Validity checks on the input
	\item Effect of parameters
\enumend

\subsection{External Interfaces}
Detailed description of all inputs and outputs
\enumstart
	\item Description of purpose
	\item Source of input, destination of output
	\item Valid range, accuracy, tolerance
	\item Units of measure
	\item Relationships to other inputs/outputs
	\item Screen and window formats
	\item Data and command formats
\enumend

\subsection{Performance}
Static numerical requirements
\enumstart
	\item Number of terminals supported
	\item Number of simultaneous users supported
	\item Amount of information handled
\enumend
Dynamic numerical requirements
\enumstart
	\item Number of transactions processed withing certain time periods (average and peek)
\enumend

\subsection{Constraints (Pseudo Requirements)}
\enumstart
	\item Is a nonfunctional requirement
	\item Standards compliance - Report format, audit tracking, etc.
	\item Implementation requirements
	\enumstart
		\item Tools, programming language, etc.
		\item Development technology and methodology should not be constrained by the client!
	\enumend
	\item Operations requirements - Administration and management of the system
	\item Legal requirements - Licensing, regulation, certification, maintenance
\enumend

\subsection{Quality Criteria for Requirements}
\enumstart
	\item Correctness - Requirements represent the clients view
	\item Completeness - All possible scenarios are described including exceptional behaviour
	\item Consistency - Requirements do not contradict each other
	\item Clarity (Unambiguity) - Requirements can be interpreted in only one way
	\item Realism - Requirements can be implemented and delivered
	\item Verifiability - Repeatable tests can be designed to show that the system fulfills the requirements
	\item Traceability - Each feature can be traced to a set of functional requirements
\enumend
