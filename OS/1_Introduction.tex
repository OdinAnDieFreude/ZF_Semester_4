\section{Introduction to operating systems}

\subsection{Goals}
\enumstart
	\item Reliability: does it keep working? \arrow availability
	\item Security: can it be compromised?
	\item Portability: how easy can it be retargeted?
	\item Performance: how fast/cheap/hungry is it?
	\item Adoption: Will people use it?
\enumend

\subsection{Roles}
\enumstart
	\item Referee: Ressource manager
	\enumstart
		\item Sharing: multiplexing hardware among applications
		\item Protection: ensure one application can't read/write/use anothers data/resources
		\item Communication: protected application must communicate
		\item Fairness: No starvation, every application makes progress
		\item Efficiency: Best use of resources, minimize power consumption
		\item Predictability: Guarantee real time performance
	\enumend
	\item Illusionist: Virtualization of resources
	\enumstart
		\item How?
		\enumstart
			\item Multiplexing: divide resources among clients
			\item Emulation: create the illusion of a resource
			\item Aggregation: join multiple resources together
		\enumend
		\item Why?
		\enumstart
			\item Sharing: enable multiple clients of a resource
			\item Sandboxing: prevent a client from accessing others resources
			\item Decoupling: avoid tying a client to a particular instance of a resource
			\item Abstraction: make a resource easier to use
		\enumend
	\enumend
	\item Glue
	\enumstart
		\item Provide high-level abstractions
		\item Extend hardware with additional functionality
		\item Hides detail
	\enumend
\enumend

\subsection{Operating system structure}
\enumstart
	\item Architectural software modes
	\enumstart
		\item Privileged mode (kernel mode)
		\enumstart
			\item Protecting the OS from applications
		\enumend
		\item User mode
	\enumend
	\item General OS structure
	\enumstart
		\item Application uses system libraries
		\item System libraries use system calls to communicate with the kernel
		\item The kernel manages ressources, \ddd
	\enumend
	\item Kernel: software that runs in privileged mode
	\enumstart
		\item Large parts of Windows, Linux (monolithic kernel)
		\item Small parts of L4, Barrelfish
		\item Just a special computer program
		\item Typically event-driven
		\item Responds to multiple entry points
		\enumstart
			\item System calls
			\item Hardware interrupts
			\item Program traps
		\enumend
		\item May include internal threads
	\enumend
	\item System libraries
	\enumstart
		\item Convenience functions
		\item System call wrappers
	\enumend
	\item Daemons: processes which are part of the OS
	\item When is the kernel entered?
	\enumstart
		\item Startup
		\item Exception: caused by user program
		\item Interrupt: caused by something else
		\item System calls
	\enumend
	\item System calls
	\enumstart
		\item RPC to the kernel
		\item Kernel is a series of syscall event handlers
		\item Mechanisms is hardware dependant
	\enumend
	\item When is the kernel exited
	\enumstart
		\item Creating a new process
		\item Resuming a process after a trap
		\item User-level upcall
		\item Switching to another process
	\enumend
\enumend
