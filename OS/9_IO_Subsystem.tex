\section{I/O subsystems}

\subsection{Partitions}
\enumstart
	\item Multiplex single disk among $>1$ file systems
	\item Contiguous block ranges per file system
\enumend

\subsection{Logical volumes}
\enumstart
	\item Emulate 1 virtual disk from $>1$ physical ones
	\item Single file system spanning $>1$ disk
\enumend

\subsection{Name files in multiple file systems}
\enumstart
	\item Top level volume names
	\item Bind "mount points" in name space
\enumend

\subsection{Virtual file systems (VFS)}
\enumstart
	\item Provides an object oriented way of implementing file systems
	\item Allow the same system calls to be used for different file systems
	\item The API is to the VFS interface
\enumend

\subsection{Interrupts}
\enumstart
	\item CPU interrupt-request line triggered by I/O device
	\item Interrupt handler receives interrupts
	\item Maskable: to ignore or delay some interrupts
	\item Interrupt vector to dispatch interrupt to correct handler (priorities, non maskable/maskable)
	\item Mechanism also used for exceptions
\enumend

\subsection{I/O protection}
\enumstart
	\item Can be dangerous to normal system operations
	\item I/O operations usually privileged
	\item Performed via system calls
	\item DMA transfers must be carefully checked
\enumend

\subsection{Device drivers}
\enumstart
	\item Software object, which abstracts a device
	\item Sits between hardware and OS
	\item Presents uniform interface to the rest of the OS
	\item Device abstractions vary
	\item Terminology
	\enumstart
		\item 1$^{st}$ level interrupt handler (FLIH) (linux: top-half)
	\enumend
	\item Problem
	\enumstart
		\item Hardware is interrupt driven
		\item Applications are often blocking
		\item Considerable processing inbetween
	\enumend
	\item Solutions
	\enumstart
		\item Driver threads
		\item Deferred procedure call (DPC)
		\enumstart
			\item Aka 2$^{nd}$ level interrupt handler, soft interrupt handler, slow interrupt handler, bottom-half handler
			\item Solution in most versions of Unix
			\item Don't need kernel threads
			\item Saves a context switch
		\enumend
		\item Demux early, run in user space
	\enumend
\enumend

\subsection{The I/O subsystem}
\enumstart
	\item Device drivers move data to and from I/O devicesy
	\item Caching: fast memory holding copy of data
	\item Spooling: hold output for a device
	\item Scheduling: some I/O ordering via per-device queue (we want fairness)
	\item Buffering: store data in memory while transferring between devices or memory
	\item Naming or discovery
	\enumstart
		\item Discovery
		\item Hotplug/-unplug
		\item Resource allocation
		\item Match driver code to device
		\enumstart
			\item Devices have unique identifiers
			\item Drivers recognize particular identifiers
			\item Kernel offers a device to each driver in turn
			\item Unix
			\enumstart
				\item Kernel creates identifiers for character devices, block devices, \ddd
				\item Major device number: class of devices
				\item Minor device number: specific device
			\enumend
		\enumend
		\item Name devices inside the kernel
		\item Name devices outside the kernel
	\enumend
	\item Unix block devices
	\enumstart
		\item Used for structured I/O
		\item Often look like files
	\enumend
	\item Character devices
	\enumstart
		\item Used for unstructured I/O
		\item Buffering implemented by libraries
		\item Examples: keyboard, mouse, serial line, \ddd
		\item Destinction with block devices somewhat arbitrary
	\enumend
	\item Naming devices outside the kernel
	\enumstart
		\item Device files (special files, crated with mknod() syscall)
	\enumend
	\item Pseudo devices in unix
	\enumstart
		\item Devices with no hardware
	\enumend
	\item Old style unix device configuration
	\enumstart
		\item All drivers compiled into the kernel
		\item Each driver probes for any supported devices
		\item System administrator populates /dev
	\enumend
	\item Linux device configuration today
	\enumstart
		\item Physical hardware configuration readable from /sys
		\item Drivers dynamically loaded as kernel modules 
		\item /dev populated dynamically by udev
	\enumend
\enumend

\subsection{Interface to network I/O}

\subsubsection{Routing}
\enumstart
	\item OS protocol stacks include routing functionality
	\item Routing protocol typically in a user-space daemon
	\item Forwarding information typically in kernel
\enumend

\subsubsection{Network stack implementation}
\enumstart
	\item One of the most important peripheral
	\item Disk interface looks increasingly like a network
	\item Step-by-step
	\enumstart
		\item Interrupt: allocate buffer, enqueue packet, \ddd
		\item Software interrupt: high priority, defragmentation, TCP processing, enqueue on socket, \ddd
		\item Application: copy buffer to user space, application processes context
	\enumend
	\item In-kernel protocol graph
	\enumstart
		\item Nodes can be per-protocol or per-connection
	\enumend
\enumend

\subsection{Memory management}
\enumstart
	\item How to ship data around?
	\item Add/remove header must be easy
	\item Avoid copying
	\item Uniformly refer to half-defined packets
	\item Fragment large datasets into smaller units
	\item Solution: data is held in a linked list of "buffer structures" (bsd: mbuf, linux: sk\_buff)
\enumend

\subsubsection{Performance}
\enumstart
	\item Put TCP procession into hardware
	\item Buffering: transfer lots of packets in a single transition
	\enumstart
		\item Decouple sending and receiving
		\item Batch together requests
		\item Producer-consumer buffer descriptor rings
	\enumend
	\item Interrupt coalescing / throttling
	\enumstart
		\item Don't interrupt on every packet
		\item Don't interrupt at all, if load is very high
	\enumend
	\item Receive-side scaling: parallelize \arrow direct interrupts and data to different cores
	\enumstart
		\item Insights: too much traffic for one core to handle
		\item Cores aren't getting faster
		\item Handle different flows on different cores
		\item Demultiplexing on the NIC (\arrow use DMA)
		\item Can balance flow across cores
	\enumend
\enumend
