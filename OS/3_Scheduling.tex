\section{Scheduling}
\enumstart
	\item Deciding, how to allocate a single ressource among multiple clients
	\item Metrics
	\enumstart
		\item Fairness
		\item Policy
		\item Balance
		\item Power
	\enumend
	\item Scheduling algorithm complexity tradeoff
	\enumstart
		\item More scheduler complexity/overhead \arrow better scheduling
		\item Less scheduler complexity/overhead \arrow worse scheduling
	\enumend
	\item Scheduling frequency
	\enumstart
		\item Higher context switch rate \arrow lower throughput
		\item Lower context switch rate \arrow better responsiveness
	\enumend
	\item Types of jobs
	\enumstart
		\item Batch workloads (run job to completion and signal when done)
		\item Interactive workloads (wait for external event and react fast)
		\item Soft realtime workloads (deadlines, guarantees)
		\item Hard realtime workloads (mission-critical, extremely time-sensitive)
	\enumend
	\item Assumptions and definitions
	\enumstart
		\item Only one CPU
		\item Fixed speed CPU
		\item We only consider work conservatice scheduling (no processor is idle, if there is a runnable task)
		\item The system can always preempt a task
	\enumend
	\item When to schedule
	\enumstart
		\item A running process blocks
		\item A blocked process unblocks
		\item A running or waiting process terminates
		\item An interrupt occurs
	\enumend
	\item Preemption
	\enumstart
		\item Non-preemptive scheduling (requires each process to explicitly give control to the scheduler)
		\item Preemptive scheduling (OS interrupts each process)
		\item Soft-realtime systems are usually preemptive
		\item Hard-realtime systems are orften not
	\enumend
	\item Overhead
	\enumstart
		\item Dispatch latency
		\item Scheduling cost
		\item Time slice allocated to a process should be significantly more than scheduling overhead
		\item Tradeoff: response time vs scheduling overhead
	\enumend
	\item Batch oriented scheduling
	\enumstart
		\item First-come first-served
		\enumstart
			\item Small jobs can wait for a long time \arrow worsening average waiting time
		\enumend
		\item Shortest job first
		\enumstart
			\item Minimizes waiting time \arrow optimal
			\item Long jobs can starve
		\enumend
		\item Shortest remaining time next
		\enumstart
			\item Not always possible (e.g. printer)
			\item Long jobs can starve
		\enumend
	\enumend
	\item Scheduling interactive loads
	\enumstart
		\item Round-robin
		\item Priority based scheduling
		\enumstart
			\item Dispatch highest priority runnable task (priority queue)
			\item Same priority \arrow other scheduling algorithm (e.g. round-robin)
			\item Starvation \arrow low priority tasks may starv
			\item Ageing (gradually increase priority for waiting tasks) \arrow solves starvation
			\item Multilevel feedback Queues
			\enumstart
				\item Penalize CPU-bound tasks to benefit I/O bound tasks \arrow I/O-tasks block before they use their quantum
			\enumend
		\enumend
	\enumend
	\item Real-time scheduling
	\enumstart
		\item Real-time based guarantees to tasks
		\enumstart
			\item Tasks appear at any time
			\item Tasks can have deadlines
			\item Execution time is generally known
			\item Periodic or aperiodic tasks
			\item Unscheduable tasks must be rejected
		\enumend
		\item Rate-monotonic scheduling
		\enumstart
			\item periodic tasks \arrow schedula task with shortest period first
			\item $m$ tasks, $C_i =$ execution time of $i$-th task, $P_i =$ Period of the $i$-th task
			\item RMS will find a schedule if: $\sum_{i=1}^m\frac{C_i}{P_i} \le m(2^{\frac{1}{m}}-1)$
		\enumend
		\item Earliest deadline first
		\enumstart
			\item Task do not need to be periodic
			\item More complex $O(n)$ for scheduling decisions
			\item EDF will find a schedule if: $\sum_{i=1}^m\frac{C_i}{P_i} \le 1$
		\enumend
	\enumend
	\item Multiprocessor scheduling
	\enumstart
		\item Problem: move between cores
		\enumstart
			\item Tend to move between caches
			\item Really bad locality \arrow bad performance
			\item Solution
			\enumstart
				\item Affinity scheduling
				\item Hierarchical scheduling
			\enumend
		\enumend
		\item Problem: parallel applications
		\enumstart
			\item Global barriers \arrow one slow thread lowers performance of application
			\item Cache sharing would be great for threads
			\item Solution: co-scheduling (schedule threads af an application together)
		\enumend
	\enumend
	\item Multicore scheduling
	\enumstart
		\item Two-dimensional: when? on which core? \arrow NP-complete
		\item Littles law
	\enumend
\enumend
