\section{File system implementation}
\enumstart
	\item Directory implementation
	\enumstart
		\item Linear list
		\item Hash table
		\item B-tree
	\enumend
	\item File type
	\enumstart
		\item Executable file
		\item Directories, symbolic link
		\item Text and binary
		\item I/O devices
		\item Named pipes (FIFOs)
		\item UNIX domain sockets
	\enumend
	\item Open file interface
	\enumstart
		\item Byte sequence (vector of bytes)
		\item Record sequence (fixed records)
		\item Key-based, tree structured
	\enumend
	\item Random access
	\enumstart
		\item Read
		\item Write
		\item Seek: absolut or relative to current position
		\item Tell: return current index
		\item State: current position in the file
	\enumend
	\item Memory mapped files: map file content into virtual address space
	\item On-disk data structures
	\enumstart
		\item Disk addressing
		\enumstart
			\item Tracks, sectors, spindles, \ddd
			\item Bad sector maps
			\item Logical block addressing: linear array of usable blocks
			\item Abstract other block storage devices (Flash-drives, SAN, \ddd)
		\enumend
	\enumend
	\item FAT32
	\enumstart
		\item No access control
		\item Very little metadata
		\item Limited volume size
		\item No support for hardlinks
		\item extensively used
		\item Files implemented as linked lists in FAT
	\enumend
	\item Unix fast file system (FFS)
	\enumstart
		\item Inode array
		\item Inode (1 block = 4096 bytes): metadata + block pointers
		\item Inode metadata: 512 bytes
		\item Block addresses: 8 bytes
		\item 448 block pointers
		\item Single-, double-, triple indirect block
		\item Free space map (bitmap, 1 bit per block) \arrow fast
	\enumend
	\item Optimization
	\enumstart
		\item Keeping together related data (files, metadata, free space map, directories, \ddd)
		\item First-fit allocation to improve disk locality
		\item Layout and block groups defined in the superblock, replicated several times
	\enumend
	\item New technology file system (NTFS)
	\enumstart
		\item NTFS master file table (1 kB Entries)
		\item Small files: small files fit into MFS records
		\item Hard links stored in MFT
		\item Normal files: MFT entry holds list of extents (Start, length)
		\item Too many attributes: Attribute list (another MFT entry) holds list of attribute locations
		\item File system metadata is hold in files
	\enumend
	\item In-memory data structures
	\enumstart
		\item Opening a file: directories are translated into kernel data structures on demand
		\item Reading and writing: Per process open file table \arrow System open file table \arrow cache of inodes
	\enumend
	\item Efficiency depends on
	\enumstart
		\item Disk allocation and directory algorithm
		\item Type of data kept in file's directory entry
	\enumend
	\item Performance
	\enumstart
		\item Disk cache
		\item Free-behind and read-ahead: techniques to optimize sequential access
		\item Dedicating section of memory as virtual disk or RAM disk
	\enumend
	\item Page cache
	\enumstart
		\item Page cache rather than disk blocks
		\item Memory-mapped I/O uses a page cache
		\item Routine I/O through the file system uses disk cache
	\enumend
\enumend
