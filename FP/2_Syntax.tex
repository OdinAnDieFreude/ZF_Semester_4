\section{Syntax}
\enumstart
	\item Naming
	\enumstart
		\item Functions start with a lowercase letter
		\item Types start with an uppercase letter
	\enumend
	\item Language constructs
	\enumstart
		\item Function application
		\enumstart
			\item Function application by whitespace: $\langle$function$\rangle$ $\langle$argument$\rangle$
			\item Binary functions can be used infix by using backticks: $v_1$ \backtick function\backtick $v_2$
			\item An infix binary relation is called an operator
			\item An operator can be written prefix: (operator) $v_1 v_2$
			\item Operators have different binding strengths
		\enumend
		\item Guards
		\enumstart
			\item xor $x$ $y$\\ $.\ \ \ | x ==$ True $=$ not $y$\\ $.\ \ \ | x ==$ False $= y$
		\enumend
		\item Pattern matching
		\enumstart
			\item xor True $y$ = not $y$\\xor False $y$ = $y$
			\item Recursive definition: pattern $p$ matches term $a$ by recursion on $p$
			\enumstart
				\item Constant: $p = c$ succeeds if $c = a$
				\item Variable: $p = x$ succeeds and with binding $x = a$
				\item Wild card: $p = \_$ succeeds with no binding
				\item Tuple: $p = (p_1, \mathellipsis, p_k)$ succeeds if $a = (a_1, \mathellipsis, a_k)$ and $p_i$ matches $a_i$ for $i \in \{1, \mathellipsis, k\}$
				\item List: $p = (p_1 : p_2)$ succeeds if $a$ is a nonempty list $(a_1 : a_2)$ and $p_1$ mathces $a_1$ and $p_2$ matches $a_2$
			\enumend
		\enumend
		\item List comprehension
		\enumstart
			\item $[X | A_1, \mathellipsis, A_k]$
			\item $X$ is an expression
			\item $A_i$ is either a Predicate or an assignment
			\item Assignments have the form $v \leftarrow [a_1, \mathellipsis, a_k]$
			\item Pattern-matching possible
		\enumend
		\item Global scope functions
		\enumstart
			\item f $x$ $y$ = $\mathellipsis$
		\enumend
		\item Local scope functions
		\enumstart
			\item Let (builds one expression from others)
			\enumstart
				\item let x1 = e1\\$\vdots$\\.$\ \ \ \ $xn = en\\in e
			\enumend
			\item Where
			\enumstart
				\item f p1 p2 $\mathellipsis$ pm\\.$\ \ \ |$ g1 = e1\\$\vdots$\\.$\ \ \ |$ gk = ek\\.$\ \ \ $where\\.$\ \ \ \ \ $v1 a1 $\mathellipsis$ an = r1\\.$\ \ \ \ \ $v2 = r2
			\enumend
		\enumend
		\item The type keyword
		\enumstart
			\item <TODO>
		\enumend
		\item $\lambda$-expressions
		\enumstart
			\item $\backslash a_1 \mathellipsis a_k \rightarrow$ expression
			\item Create functions inline
			\item $a_i$ are patterns
			\item the functions takes the input values defined in the patterns and evaluates to whatever the expression says
		\enumend
	\enumend
\enumend

\subsection{Type classes}
\enumstart
	\item Type classes are a way to implement polymorphism in haskell
	\item A type can be an instance of multiple typeclasses
	\item Polymorphic arguments of functions can be restricted to members of certain typeclasses
	\item Definition
	\enumstart
		\item class Eq a where$\\. \ \ \ (==), (/=) :: a \rightarrow a \rightarrow Bool$\\$. \ \ \ x /= y = $not$(x==y)$
		\item Definition contains
		\enumstart
			\item Class name
			\item Signature: List of function names and types
			\item (Optional standard-)definitions: can be overwritten
		\enumend
	\enumend
	\item Derived classes
	\enumstart
		\item class Eq a $\Rightarrow$ Ord a where $\mathellipsis$
	\enumend
	\item Instantiation
	\enumstart
		\item instance Eq Bool where\\True == True = True\\False == False = True\\ \_ \_ = False
	\enumend
\enumend
